\chapter{Les enjeux d'une organisation collaborative}

Pour réaliser des produits et des services, une entreprise a besoin d'une collaboration optimale entre le client et le prestataire, entre les équipes de travail, ainsi qu'entre les membres au sein de chaque équipe. Nous verrons dans cette première partie quels sont les facteurs qui jouent sur la qualité d'une telle collaboration, ainsi que ses différents enjeux.

\section{Niveaux de collaboration}

Il est important de dissocier l'efficacité de l'équipe et l'efficacité de ses membres. En outre, il faut savoir que l'efficacité de l'équipe n'est pas nécessairement égale à la somme de l'efficacité de ses membres.

En gestion de projet, il existe la notion de \textit{mois / homme}, qui est une unité visant à désigner la quantité de travail nécessaire à la réalisation d'une tâche dans un projet. Dans \textit{Le mythe du mois homme}, Frédéric Brooks nous explique que cette notion de mois/homme est quelque peu arbitraire. En effet, on pourrait penser que pour qu'un projet se réalise deux fois plus vite, il suffirait de doubler l'effectif alloué à ce dernier. Hors les tâches ne sont pas décomposables et parallélisables en autant de sections que souhaité. Brooks illustre ces propos par une phrase qui est devenue une plaisanterie célèbre en gestion de projet : \begin{Quote}Neuf femmes ne peuvent pas mettre au monde en un enfant un mois.\end{Quote}

On remarque toutefois que cette observation est fonction des outils utilisés pendant la réalisation du projet. En effet, divers logiciels facilitent la création de contenu à plusieurs personnes : ils favorisent ainsi la parallélisation de certaines tâches. Il ne faut donc pas négliger l'utilisation de ces outils, que nous approfondirons dans la troisième partie de ce dossier.

Ainsi, en terme de performance, la réussite de la collaboration de l'équipe peut être définie par le ratio : \[\text{\textit{efficacité de l'équipe}} \over \text{\textit{efficacité de ses membres}}\]

On peut dès lors observer trois niveaux de collaboration :\\

\begin{itemize}

\item \textbf{Efficacité de l'équipe \textless \: efficacité de ses membres} \footnote{Nous entendons par \textit{efficacité des membres}, l'efficacité de la somme de tous les membres de l'équipe.} (faible collaboration) :
Le groupe est moins performant qu'il le pourrait. Il n'est peut-être pas assez soudé ou certains éléments troublent le travail de l'équipe, ou alors l'organisation et la gestion du travail de l'équipe ne sont pas assez efficace et engendrent des pertes de productivité.\\

\item \textbf{Efficacité de l'équipe = efficacité de ses membres} (bonne collaboration) :
Les membres du groupe s'entendent suffisamment bien pour travailler ensemble dans de bonnes conditions. Le groupe peut faire face à certains problèmes qui pourrait perturber le travail, il y a plusieurs interactions positives et la gestion du travail est convenable.\\

\item \textbf{Efficacité de l'équipe \textgreater \: efficacité de ses membres} (excellente collaboration) :
Schéma idéal et particulièrement difficile à atteindre, il est toutefois possible d'y parvenir grâce à une équipe particulièrement soudée et organisée. Il est également nécessaire de favoriser l'échange de connaissances entre les différents membres de l'équipe et d'utiliser des outils et des méthodes de travail collaboratif adaptés, que nous étudierons prochainement.

\end{itemize}

\subsection{Loi de Brooks}
Frederic Brooks, dans ses recherches, fait une prédiction sur la productivité des projets informatiques, appelée \textit{loi de Brooks} :

\begin{Quote}
Ajouter des personnes à un projet en retard accroît son retard.
\end{Quote}

Il va de soit que cette observation dépend du contexte, elle peut se révéler invalide dans certaines situations. Toutefois, pour l'énoncer, Brook se base sur deux postulats :
\begin{itemize}
\item La plupart des tâches d'un projet ne sont pas partitionnables ;
\item les nouveaux arrivants vont faire perdre du temps aux équipes en place.
\end{itemize}

Nous avons déjà cité le premier d'entre eux dans le précédent chapitre. Concernant le deuxième postulat, le temps perdu est proportionnel à $ n(n-1) $ (où n est le nombre de personnes impliquées).

La taille d'une équipe influe donc dans la productivité. Par exemple, en étant 10 dans une cuisine, il y a peu de chance de pouvoir faire un aussi bon travail qu'à trois personnes. En effet, les personnes pourraient se gêner dans cet espace réduit, se disputer les tâches a effectuer, attendre la disponibilité des ustensiles de cuisine.

En entreprise, un temps de formation et de compréhension est nécessaire avant que les employés s'impliquent dans un projet. D'autre part, ajouter des personnes à une équipe augmente de manière exponentielle le nombre des canaux de communication entre chaque personne \figref{canaux_comm}. Il y a un donc un certain équilibre à trouver dans l'effectif d'une équipe, afin d'optimiser son efficacité.

Ce nombre dépend bien évidement de plusieurs paramètres contextuels, tels que le type de projet ou les capacités de collaboration. Kurt Lewin \footnote{Kurt Lewin (1890-1947) : psychologue américain spécialisé dans la psychologie sociale et le comportementalisme, acteur majeur du mouvement \textit{l'école des relations humaines}}, dans ses recherches, affirme qu'un groupe de 6 à 10 personnes favorise les échanges, car un bon équilibre s'instaure entre le dynamisme du groupe, le temps et la richesse de la production.

\fig{img/canaux_comm.jpg}{canaux_comm}{Schéma représentant le nombre de canaux de communication d'un groupe de personnes, illustré par un réseau téléphonique maillé.}

\begin{app}
Dans l'entreprise GFI, les équipes de recherche sont très petites par rapport aux autres équipes (1 ou 2 personnes), alors que le client attend des résultats dans les plus brefs délais. Les chefs de projets considèrent qu'augmenter la taille de l'équipe ne ferait pas gagner de temps, car celles-ci travaillent sur des projets dont la formalisation des besoins varie régulièrement.
\end{app}

\section{Vers une collaboration réussie}

On peut citer les différents facteurs qui favorisent la collaboration dans une équipe de travail, permettant d'atteindre le troisième niveau de collaboration évoqué précédemment.

\begin{itemize}
\item \textbf{La motivation à travailler} : Elle peut être due aux conditions de travail ou au stress. En outre, si une personne manque de motivation, elle peut s'écarter du groupe (qui lui, désire peut-être travailler), ce qui nuit à la collaboration de l'équipe.

\item \textbf{L'intérêt accordé au projet} : Si une personne aime son travail, le résultat en sera très satisfaisant. Elle cherchera a en comprendre les enjeux, ce qui favorisera les dialogues dans l'équipe.

\item \textbf{L'esprit d'équipe} : Enfin, l'esprit d'équipe est une notion fondamentale dans la collaboration. Aujourd'hui, les entreprises se focalisent sur le gain pouvant être apporté par une équipe soudée : on notamment remarquer la présence d'activités collectives en entreprise. Une expérience réalisée par \textit{l'école de Chicago} a démontré que les interactions entre les individus sont plus importantes encore que la motivation et l'intérêt de chaque membre. On peut également noter que certaines méthodes de travail favorisent l'esprit d'équipe, comme nous le verrons dans la prochaine partie de ce rapport.
\end{itemize}
~\\

De nos jours, un chef de projet, en plus de veiller au bon déroulement du projet, doit prendre en compte ces observations afin de garantir une bonne productivité.

\section{Structure des équipes de travail}

Nous avons vu qu'augmenter la taille de l'équipe augmente la difficulté de la collaboration. Il s'agissait ici d'une approche théorique, démontrée mathématiquement \figref{canaux_comm}. Néanmoins, avec une approche pratique, on peut constater que la communication ne se fait pas de manière égale entre chaque personne. Certains canaux sont privilégiés : entre deux personnes qui s'entendent bien, mais aussi lorsqu'ils font intervenir certains individus, comme le chef de projet ou l'animateur de réunion. En passant par un ou plusieurs intermédiaires, nous réduisons ainsi le nombre d'échanges entre les membres de l'équipe et la collaboration devient plus efficace.\\

L'instauration d'une hiérarchie au sein de l'équipe peut donc se révéler nécesaire. Lorsque la réalisation d'un projet fait intervenir un grand nombre de personnes, le découpage des effectifs en équipes et en sous-équipes est nécessaire. Nous pouvons ainsi garantir une taille raisonnable pour chaque groupe de travail.

Ainsi, la structuration des équipes de travail n'est pas une tâche à sous-estimer car elle va influencer la réussite de la collaboration. Elle peut être représentée sous la forme d'un organigramme \figref{orga_hierarchique}. Ici, l'équipe \textit{Production du produit A} (en violet) devra passer par le \textit{Service production} (en jaune), pour communiquer avec l'équipe \textit{Production du produit B}.\\

Il faut toutefois noter que les équipes de travail ne sont pas nécessairement structurée de manière hiérarchique, elles peuvent l'être aussi de manière matricielle \figref{orga_matriciel}.

\fig{img/orga_hierarchique.png}{orga_hierarchique}{Organigramme hiérarchique d'une société produisant trois types de produit.}

Une structure matricielle est envisageable lorsque nous pouvons associer aux équipes de travail certaines thématiques (ici : \textit{production, finance} et \textit{DRH}) : elle est donc adapté à des situations particulières. Cette structure favorise les échanges transversaux entre les équipes, et favorise ainsi la collaboration entre les équipes qui travaillent sur les mêmes problématiques. Elle est toutefois plus dure à mettre en œuvre : l'entreprise concernée fait ainsi le choix d'opter pour une organisation plus difficile, mais au profit d'une meilleure collaboration.

\smallfig{img/orga_matriciel.png}{orga_matriciel}{Organigramme matriciel de la même société.}

\begin{app}
L'organigramme de l'entreprise GFI se présente sous la forme matricielle. Cela favorise le dynamisme et les échanges entre chaque équipe de travail.
\end{app}

\section{Conclusion}

Nous avons vu les différents enjeux d'une organisation collaborative du travail. Nous pouvons en déduire que le facteur humain et les interactions entre les individus sont importants. Ce sont aujourd'hui les enjeux de notre société, en opposition aux enjeux techniques du temps travail à la chaîne.
