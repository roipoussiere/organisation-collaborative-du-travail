\chapter{Les enjeux d'une organisation collaborative}

...Introduction...

\section{Les gains de productivité}

...

\section{Efficacité de la collaboration}

Pour réaliser des produits et des services, une entreprise a besoin d'une collaboration optimale entre le client et le prestataire, entre les équipes de travail, ainsi qu'entre les membres au sein de chaque équipe.

En outre, la qualité de cette collaboration dépend de plusieurs paramètres tels que la personnalité des individus, les conditions de travail, l'entente au sein de l'équipe, etc.

Il est important de dissocier l'efficacité de l'équipe et l'efficacité de ses membres. En outre, il faut savoir que l'efficacité de l'équipe n'est pas nécessairement égale à la somme de l'efficacité de ses membres.

Dans \textit{Le mythe du mois homme}, Frédéric Brooks nous explique que la notion de l'unité mois/homme, utilisée en gestion de projet pour désigner la quantité de travail nécessaire à la réalisation d'une tâche, est quelque peu arbitraire. En effet, on pourrait penser que pour qu'un projet se réalise deux fois plus vite, il suffirait de doubler l'effectif alloué à ce dernier. Hors les tâches ne sont pas décomposables et parallélisables en autant de sections que souhaité. Brooks illustre ces propos par une phrase qui est devenue une plaisanterie célèbre en gestion de projet : \begin{Quote}Neuf femmes ne peuvent pas mettre au monde un enfant un mois.\end{Quote}

On remarque toutefois que cette observation est fonction des outils utilisés pendant la réalisation du projet. En effet divers logiciels facilitent la création de contenu à plusieurs personnes : ils favorisent ainsi la parallélisation de certaines tâches. Il ne faut donc pas négliger l'utilisation de ces outils, que nous approfondirons dans la troisième partie de ce dossier.

Ainsi, en terme de performance, la réussite de la collaboration de l'équipe peut être définie par le ratio : \[\text{\textit{efficacité de l'équipe}} \over \text{\textit{efficacité de ses membres}}\]

On peut dès lors observer trois niveaux de collaboration :\\

\begin{itemize}

\item \textbf{Efficacité de l'équipe \textless \: efficacité de ses membres} (faible collaboration) :
Le groupe est moins performant qu'il le pourrait. Il n'est peut-être pas assez soudé ou certains éléments troublent le travail de l'équipe, ou alors l'organisation et la gestion du travail de l'équipe ne sont pas assez efficace.\\

\item \textbf{Efficacité de l'équipe = efficacité de ses membres} (bonne collaboration) :
Les membres du groupe s'entendent suffisamment bien pour travailler ensemble dans de bonnes conditions. Le groupe peut faire face à certains problèmes qui pourrait perturber le travail, il y a plusieurs interactions positives et la gestion du travail est convenable.\\

\item \textbf{Efficacité de l'équipe \textgreater \: efficacité de ses membres} (excellente collaboration) :
Schéma idéal et particulièrement difficile à atteindre, il est toutefois possible d'y parvenir grâce à une équipe particulièrement soudée et organisée. Il est également nécessaire de favoriser l'échange de connaissances entre les différents membres de l'équipe et d'utiliser des outils de travail collaboratif adaptés.

\end{itemize}

\subsection{Loi de Brooks}
Le livre de Brooks a donné naissance à \textit{la loi de Brooks}, qui est une prédiction sur la productivité des projets informatiques :

\begin{Quote}
Ajouter des personnes à un projet en retard accroît son retard.
\end{Quote}

Pour l'énoncer, Brook se base sur deux postulats :
\begin{itemize}
\item La plupart des tâches d'un projet ne sont pas partitionnables ;
\item les nouveaux arrivants vont faire perdre du temps aux équipes en place.
\end{itemize}

Nous avons déjà étudié le premier d'entre eux dans le précédent chapitre. Concernant le deuxième postulat, le temps perdu est proportionnel à $ n(n-1) $ (où n est le nombre de personnes impliquées).

La taille d'une équipe influe donc dans la productivité. Par exemple, en étant 10 dans une cuisine, il y a peu de chance de pouvoir faire un aussi bon travail qu'à trois personnes. En entreprise, un temps de formation et de compréhension est nécessaire avant que les employés s'impliquent dans un projet. En outre, ajouter des personnes à une équipe augmente de manière exponentielle le nombre des canaux de communication entre chaque personne \figref{canaux_comm}. Il y a un donc un certain équilibre à trouver dans l'effectif d'une équipe afin d'optimiser son efficacité. Ce nombre dépend bien évidement du contexte (type de projet, capacités de collaboration, etc.).

\fig{img/canaux_comm.jpg}{canaux_comm}{Schéma représentant le nombre de canaux de communication d'un groupe de personnes, illustré par un réseau téléphonique maillé.}

\begin{app}
Dans le cas de GFI, les équipes de recherche sont très petites par rapport aux autres équipes (1 ou 2 personnes), alors que le client attend des résultats dans les plus brefs délais. Les chefs de projets considèrent qu'augmenter la taille de l'équipe ne ferait pas gagner de temps, car celles-ci travaillent sur des projets dont la formalisation des besoins varient régulièrement.
\end{app}

\section{L'organisation en dehors de l'entreprise}

On peut noter l'existence de projets réalisés par de nombreuses personnes, sans aucun rapport avec une quelconque entreprise. Par exemple le développement du noyaux Linux (un système d'exploitation open-source), est réalisé par des milliers de développeurs de par le monde, de manière indépendante et sans qu'aucune entité, entreprise ou  association, ne gouverne ce travail. Il est ici évident que pour que ce projet avance, ces personnes doivent collaborer ensemble et suivre un système d'organisation adapté et rigoureux.

Le développement d'Internet a grandement facilité le partage d'information entre personnes à travers le monde. Ainsi, certains projets collaboratifs ont pu voir le jour, faisant intervenir différents acteurs géographiquement distants, sans qu'il n'y ait d'entité juridique ni d'établissement physique définis. Dans le cadre de ces projets, qui peuvent faire intervenir des milliers de personnes, une bonne collaboration est indispensable.

\section{Organisation selon la taille du projet}

Notions d'équipe, de chef de projet, hiérarchie, etc.

\section{Application}

\subsection{Le cas de GFI Informatique}

GFI Informatique est une \gls{ESN} \footnote{ESN : Société de services numériques spécialisée en génie informatique.} implantée sur de nombreux sites en France et en Europe, elle compte plus de 10 000 salariés. Les secteur d'activités développée sont la banque, industrie, secteur public, télécom, énergie et enfin transport, ce dernier étant très actif à Toulouse avec Airbus et son écosystème. Ainsi sur le site de Toulouse, comprenant près de 500 salariés, le client principal est Airbus, suivi par EDF, la mairie de Toulouse et ATR.

En règle générale, les petits clients n'imposent ni les outils ni les méthodes : ils se contentent d'expliciter les objectifs attendus, qu'importe les moyens mis en œuvre pour y parvenir.

En revanche, pour les clients plus important et principalement avec Airbus, le rapport de force entre le client et le fournisseur est inversé : les outils et méthodes à appliquer sont ici imposées. En effet, la taille importante du projet oblige le client à traiter avec plusieurs fournisseurs, d'autre part Airbus comporte également des équipes de développement en interne.

Le fait d'imposer ces outils et méthodes constituent une difficulté et peuvent même être source de conflits au niveau des équipes de travail, qui ont chacune leurs préférences. Mais l'utilisation des mêmes méthodes et des mêmes outils par tous les acteurs du projet vise toutefois à simplifier l'organisation. Ainsi, ajouter des difficultés de manière locale peut s'avérer bénéfique de manière globale.

\subsection{Le cas de Ineo}



\section{Conclusion}
