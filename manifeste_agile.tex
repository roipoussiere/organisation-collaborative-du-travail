\label{manifeste_agile}

\backchapter{Manifeste pour le développement Agile de logiciels}

Nous découvrons comment mieux développer des logiciels par la pratique et en aidant les autres à le faire.
Ces expériences nous ont amenés à valoriser :

\begin{itemize}
\item \textbf{Les individus et leurs interactions}, plus que les processus et les outils ;

\item \textbf{Des logiciels opérationnels}, plus qu’une documentation exhaustive ;

\item \textbf{La collaboration avec les clients}, plus que la négociation contractuelle ;

\item \textbf{L’adaptation au changement}, plus que le suivi d’un plan.

\end{itemize}

Nous reconnaissons la valeur des seconds éléments, mais privilégions les premiers.

\section*{Principes sous-jacents au manifeste}

Nous suivons ces principes:

\begin{itemize}

\item Notre plus haute priorité est de satisfaire le client
en livrant rapidement et régulièrement des fonctionnalités
à grande valeur ajoutée.

\item Accueillez positivement les changements de besoins,
même tard dans le projet. Les processus Agiles
exploitent le changement pour donner un avantage
compétitif au client.

\item Livrez fréquemment un logiciel opérationnel avec des
cycles de quelques semaines à quelques mois et une
préférence pour les plus courts.

\item Les utilisateurs ou leurs représentants et les
développeurs doivent travailler ensemble quotidiennement
tout au long du projet.

\item Réalisez les projets avec des personnes motivées.
Fournissez-leur l’environnement et le soutien dont ils
ont besoin et faites-leur confiance pour atteindre les
objectifs fixés.

\item La méthode la plus simple et la plus efficace pour
transmettre de l’information à l'équipe de développement
et à l’intérieur de celle-ci est le dialogue en face à face.

\item Un logiciel opérationnel est la principale mesure d’avancement.

\item Les processus Agiles encouragent un rythme de développement
soutenable. Ensemble, les commanditaires, les développeurs
et les utilisateurs devraient être capables de maintenir
indéfiniment un rythme constant.

\item Une attention continue à l'excellence technique et
à une bonne conception renforce l’Agilité.

\item La simplicité – c’est-à-dire l’art de minimiser la
quantité de travail inutile – est essentielle.

\item Les meilleures architectures, spécifications et
conceptions émergent d'équipes autoorganisées.

\item À intervalles réguliers, l'équipe réfléchit aux moyens
de devenir plus efficace, puis règle et modifie son
comportement en conséquence.

\end{itemize}