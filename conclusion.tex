\backchapter{Conclusion}

Nous avons vu au fil de ce dossier les enjeux de l'organisation collaborative du travail. Nous nous sommes ensuite intéressés aux différents outils et aux méthodes qui permettent une bonne collaboration, ainsi que la manière de les utiliser et leurs limites.\\

On peut observer chez certaines espèces animales, notamment les fourmis, d'autres formes d'organisation collaborative du travail. L'élaboration d'une fourmilière, la récolte et le stockage de la nourriture ou la protection du nid, font appel à plusieurs centaines, voir milliers d'individus qui ont une tâche définie dépendant du contexte et des saisons. Par conséquent, ces sociétés animales font preuve d'une excellente organisation pour ces projets, qui peuvent avoir comme équivalent humain la construction d'un immeuble ou la gestion d'un hôpital.

Disposent t'elles pour autant d'outils pour mener à bien cette organisation, tels que des logiciels de gestion ou des documents ? Non, bien sur. Il est peu probable également qu'elles utilisent nos méthodes de travail complexes, telles que le cycle en V ou les méthodes agiles. Même s'il serait difficile de l'affirmer, elles semblent davantage se tenir à de simples procédés plutôt que des méthodes.

On peut dès lors constater que les outils et les méthodes de travail modernes ne sont pas nécessaires a une bonne organisation, même si elles jouent un rôle important.

Mais alors d'où provient cette organisation si efficace chez ces sociétés animales, en l'absence des méthodes et outils auxquels nous nous sommes accoutumés ? Ce que nous avons observé chez les fourmis, c'est un sens de collaboration plus naturel que chez les humains : elles pensent en tant que société et non en tant qu'individu. À cela s'ajoute la présence d'excellentes capacités de communication, notamment par l'échange de phéromones.\\

La communication et la mentalité des individus joue donc un rôle très important dans l'organisation collaborative. Ainsi, s'il est utile de posséder des bons outils ou de maîtriser des bonnes méthodes de travail, encore faut-il avoir l'esprit collaboratif et disposer d'un sens de la communication, nécessaires pour les exploiter pleinement.\\

D'autre part, il faut considérer que ces sociétés sont moins évoluées technologiquement que les nôtres. De fait, les enjeux ne sont pas les mêmes puisque leurs modes de fonctionnement sont moins complexes. En outre, elles ne se posent pas la question de la motivation au travail, n'ont pas à gérer la paie des employés, ne manipulent pas les outils modernes que nous utilisons, comme les ordinateurs.

Nous pouvons donc constater que ce sont nos sociétés modernes, avec nos modes de fonctionnement complexes, qui nous ont amenés à développer des outils et des méthodes adaptées. Et ce sont peut-être ces nouveaux outils et méthodes qui vont faire évoluer de nouveau notre société : l'ordinateur a permis par exemple de résoudre de nombreux problèmes, mais a aussi instauré de nouveaux enjeux.\\

Notre société est amenée à évoluer encore. Nous pouvons dès lors se demander quels seront les enjeux de demain et quelles méthodes et outils nous allons développer pour les affronter.
