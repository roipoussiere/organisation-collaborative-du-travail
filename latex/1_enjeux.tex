\chapter{Les enjeux de la collaboration}

...Introduction...

\section{Les gains de productivité}


\section{Efficacité de la collaboration}

Pour réaliser des produits et des services, une entreprise a besoin d'une collaboration optimale entre le client et le prestataire, entre les équipes de travail, ainsi qu'entre les membres au sein de chaque équipe.

La qualité de cette collaboration dépend de plusieurs paramètres tels que la personnalité des individus, les conditions de travail, l'entente au sein de l'équipe, etc.

Il est important de dissocier l'efficacité de l'équipe et l'efficacité de ses membres. En outre, il faut savoir que l'efficacité de l'équipe n'est pas nécessairement égale à la somme de l'efficacité de ses membres. Dans *Le Mythe du Mois Homme*, de Frederick Brooks, ...

Ainsi, en terme de performance, la réussite de la collaboration de l'équipe peut être définie par le ratio : efficacité de l'équipe / efficacité de ses membres.

On peut dès lors observer trois niveaux de collaboration :

\begin{itemize}

\item \textbf{Efficacité de l'équipe < efficacité de ses membres} (faible collaboration) :
Le groupe est moins performant qu'il le pourrait. Il n'est peut-être pas assez soudé ou certains éléments troublent le travail de l'équipe, ou alors l'organisation et la gestion du travail de l'équipe ne sont pas assez efficace.

\item \textbf{Efficacité de l'équipe = efficacité de ses membres} (bonne collaboration) :
Les membres du groupe s'entendent suffisamment bien pour travailler ensemble dans de bonnes conditions. Le groupe peut faire face à certains problèmes qui pourrait perturber le travail, il y a plusieurs interactions positives et la gestion du travail est convenable.

\item \textbf{Efficacité de l'équipe > efficacité de ses membres} (excellente collaboration) :
Schéma idéal et particulièrement difficile à atteindre, il est toutefois possible d'y parvenir grâce à une équipe particulièrement soudée et organisée. Il est également nécessaire de favoriser l'échange de connaissances entre les différents membres de l'équipe et d'utiliser des outils de travail collaboratif adaptés.

\end{itemize}

\section{Quand la collaboration quitte l'entreprise}

On peut noter l'existence de projets réalisés par de nombreuses personnes, sans aucun rapport avec une quelquonque entreprise. Par exemple le développement du noyaux Linux (un système d'exploitation open-source), est réalisé par des milliers de développeurs de par le monde, de manière indépendante et sans qu'aucune entité, entreprise ou  association, ne gouverne ce travail. Il est ici évident que pour que ce projet avance, ces personnes doivent collaborer ensemble et suivre un système d'organisation adapté et rigoureux.

Le développement d'Internet a grandement facilité le partage d'information entre personnes à travers le monde. Ainsi, certains projets collaboratifs ont pu voir le jour, faisant intervenir différents acteurs géographiquement distants, sans qu'il n'y ait d'entité juridique ni d'établissement physique définis. Dans le cadre de ces projets, qui peuvent faire intervenir des milliers de personnes, une bonne collaboration est indispensable.

\section{Collaboration selon la taille du projet}


\section{Conclusion}