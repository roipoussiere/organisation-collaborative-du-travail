\backchapter{Introduction}

Lorsque un projet devient assez conséquent, sa gestion devient de plus en plus complexe. Dans le cas d'un projet informatique, le programme réalisé comportera un grande quantité de code, dont il faut faciliter la maintenance le plus tôt possible. Il fera intervenir un grand nombre de personnes, réparties dans plusieurs équipes de travail. En outre, le prestataire devra collaborer avec un client qui n'a pas les mêmes notions techniques et dont les besoins peuvent évoluer.

On peut noter toutefois l'existence de projets réalisés par de nombreuses personnes, sans aucun rapport avec une quelconque entreprise. Le développement d'Internet a grandement facilité le partage d'information entre personnes à travers le monde et certains projets collaboratifs ont ainsi pu voir le jour. Par exemple, le développement du noyaux Linux \footnote{Linux : système d'exploitation open-source.}, est développé par des milliers de personnes de par le monde, communiquant par Internet, de manière indépendante et sans qu'aucune entité, entreprise ou association, ne gouverne ce travail.

Il est ici évident que pour que ce projet avance, ces personnes doivent collaborer ensemble et suivre un système d'organisation adapté et rigoureux. Mais quels sont les secrets d'une organisation collaborative du travail efficace au sein d'un projet ?\\

Nous tenterons dans ce document de répondre à cette problématique en commençant par énoncer les différents enjeux de cette organisation collaborative. Nous nous arrêterons ensuite sur les différentes méthodes de travail qui peuvent faciliter cette collaboration, ainsi que les outils existant.

\begin{app}
Au fil du développement nous aurons une approche orientée vers le domaine de l'informatique et nous ferons régulièrement référence à des cas pratiques issus de situations existantes en entreprise, symbolisées par le sigle ci-contre.\\

L'entreprise concernée est GFI Informatique, qui est une \gls{ESN} \footnote{ESN : Société de services numériques spécialisée en génie informatique.} implantée sur de nombreux sites en France et en Europe, qui compte plus de 10 000 salariés. Les secteurs d'activités développés sont la banque, l'industrie, le secteur public, les télécoms, l'énergie et enfin le transport, ce dernier étant très actif à Toulouse avec Airbus et son écosystème. Ainsi sur le site de Toulouse, comprenant près de 500 salariés, le client principal est Airbus, suivi par EDF, la mairie de Toulouse et ATR.
\end{app}
