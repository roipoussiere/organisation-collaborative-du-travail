\chapter{Les méthodes de travail}

L'organisation d'un projet passe tout d'abord par une méthode de travail à appliquer. Son choix est généralement la première décision à prendre avec le client avant de commencer le développement d'un produit. En outre, elle aura une grande incidence sur sa réussite.

Il existe de nombreux modèles de gestion de projet dans le milieu industriel. Nous allons nous pencher dans ce chapitre sur les méthodes de travail les plus utilisées dans les projets informatiques : le cycle en V et les méthodes agiles.

\section{Le cycle en V}

En gestion de projet, il est important de considérer les erreurs comme faisant partie intégrante d'un projet (\textit{Errare Humanum est}, comme chacun sait). Dans l'industrie en général, sur le cycle de développement d'un produit, plus un problème est détecté tôt, plus il sera facile de le corriger et moins grandes en seront les conséquences. Celles-ci, appelées \textit{effets de bord}, sont dues aux causes suivantes :

\begin{itemize}
\item à la difficulté de détecter la nature du problème lorsque le produit approche de la phase opérationnelle ;
\item au coût qu'engendrera les retours en phase de développement ;
\item au coût des tâches développées après la découverte du problème, qui seront éventuellement à retravailler.
\end{itemize}

Ainsi, pendant le développement d'un produit, l'objectif n'est pas tant de limiter les erreurs, mais davantage de parvenir à les détecter le plus tôt possible \expref{rxp_tests}.\\

Le cycle en V est devenu un standard de l'industrie logicielle et dans les autres domaines de l'industrie en général. C'est un modèle de gestion de projet permettant de limiter ces effets de bord, en découpant la réalisation d'un projet en plusieurs étapes de manière séquentielle.

Le cycle en V comporte 3 phases : la conception, le développement et les tests. Les phases de conception et de tests se structurent en plusieurs étapes.\\

\fig{img/cycle_en_v.png}{cycle_v}{Schéma du cycle en V}

\begin{itemize}
\item \textbf{La phase de conception}

\begin{itemize}
\item Analyse des besoins et faisabilité ;
\item Spécification ;
\item Conception architecturale ;
\item Conception détaillée.
\end{itemize}

Les étapes de la phases de conception commencent par une approche très globale du projet, et augmentent progressivement le niveau de détail jusqu'à la phase de codage. Chaque étape de conception s'appuie sur l'étape précédente. \\

\item \textbf{La phase de développement logiciel (codage)}

Il s'agit du développement du produit, qui s'appuie sur la conception détaillée. \\

\item \textbf{La phase de tests}

\begin{itemize}
\item Tests unitaires ;
\item Tests d'intégration ;
\item Tests de validation ;
\item Recette.
\end{itemize}

\end{itemize}
~\\
On peut noter une correspondances entre les étapes situées sur le même niveau : par exemple, si il existe un problème lors d'un test d'intégration, cela va affecter la conception architecturale \figref{cycle_v}. \\

Les tests sont des étapes très importantes dans la réalisation d'un produit, car c'est ce qui permet de valider leur bon fonctionnement et la conformité aux attentes du client. D'une certaine manière, une ESN ne vend pas du code au client, mais vend les tests du code, validés. Ils commencent par un niveau de détail élevé, puis offrent une vue de plus en plus globale sur le produit final.\\

En outre, des expériences nous ont montrés que les catastrophes sont souvent déclenchées suite à plusieurs erreurs survenant en même temps. Il est donc d'usage de prendre en compte le cas le plus défavorable lors de ces tests, en se basant notamment sur la loi de Murphy : \textit{Tout ce qui peut mal tourner va mal tourner}.

\begin{app} \label{rxp_tests}
Lorsque l'entreprise GFI traite avec son client principal Airbus, elle a bien conscience de l'importance des phases de tests dans le secteur de l'aéronautique. Lorsque une anomalie se glisse dans le code, on peut en effet analyser différents scénarios :
\begin{itemize}
\item L'erreur est détectée pendant la \textbf{phase de tests unitaires} : incidence presque nulle, puisque ces tests sont exécutés de manière automatique, très régulièrement et effectués de manière très ciblée.
\item L'erreur est détectée pendant la \textbf{phase de tests d'intégration} : il y a en conséquence une légère agitation puisque il faut analyser la provenance de l'erreur, puis la transmettre à la personne ayant développé la portion de code responsable.
\item L'erreur est détectée pendant la \textbf{phase de tests d'intégration} : elle est encore plus difficile à dénicher puisque c'est le programme complet qui est testé. Les précédents tests seront à effectuer de nouveau et une partie du code sera à réécrire.
\item L'erreur est détectée \textbf{après la recette} : Airbus devra alors rapatrier tous les avions concernés et démonter certains composants afin d'effectuer un mise à jour du logiciel, ce qui aura un coût considérable.
\item L'erreur n'est \textbf{pas détectée} : En fonction de la nature de l'erreur et du moment où elle apparaît, cela peut entraîner le crash d'un avion dans des cas extrêmes.
\end{itemize}
\end{app}

Pour la réalisation d'un projet informatique, ce modèle de gestion de projet à l'avantage de prévoir et de quantifier les besoins, de choisir l'architecture logicielle à adopter, de penser à l'intégration des différentes fonctionnalités, avant de commencer son développement. Cela permet notamment d'anticiper certains problèmes de conception pouvant survenir pendant la phase de codage, et donc de réduire le développement des fonctionnalités inadaptées.

Par exemple, dans le cadre de la réalisation d'un site Internet, il sera bien utile aux développeurs de savoir que le site devra être multilingue avant de commencer la réalisation. En effet, cette fonctionnalité va influencer l'architecture générale du site, et il sera difficile d'implémenter une telle fonctionnalité en cours de développement si elle n'a pas été prévue au départ.

Ainsi, avec une approche théorique, le cycle en V possède de nombreux avantages et peut se révéler très utile dans le développement d'un projet informatique. Toutefois, la mise en pratique de ce modèle de gestion de projet a mis en valeur certains défauts.

\section{Les méthodes agiles}

La méthode du cycle en V, bien qu'elle soit intéressante d'un point de vue théorique, possède en réalité de gros inconvénients, pouvant mettre en péril la réussite du projet :

\begin{itemize}

\item Les documents de conception sont réalisés par différentes personnes qui ont chacune leur point de vue, par ailleurs il est difficile de vérifier la bonne concordance de ces différents documents. Ainsi, les développeurs peuvent se trouver face à des incohérences considérables dans les dossiers de conception.

\item Les personnes qui réalisent la conception ont souvent un point de vue trop théorique et n'ont pas forcément en tête les problèmes techniques qui pourront survenir en utilisant leurs choix de conception.

\item La rédaction des différents documents de conception prend du temps et aura donc un impact considérable sur le coût du projet.

\item En cas d'arrêt de la production, le produit est inutilisable.

\item Il est courant que le client change d'avis pendant la réalisation d'un projet. Avec le modèle du cycle en V, un tel changement impactera toutes les étapes de conception, de développement et de tests, ce qui a une forte incidence sur le coût du projet.

\item Enfin, le problème le plus important du cycle en V est l'effet dit \textit{Tunnel}. En effet, le client n'a aucune visibilité sur le projet pendant sa réalisation : il est sollicité uniquement au début (pour l'analyse des besoins) et à la fin (pour la recette). Ainsi, si une confusion apparaît sur le cahier des charges, le client s'en apercevra uniquement pendant la recette. En prenant également en compte les confusions pouvant exister entre les différentes étapes de conception, le client peut se retrouver face à un produit qui ne correspond pas du tout à ses attentes.\\

\end{itemize}

Ainsi, l'industrie informatique a parfois connu des scénarios catastrophiques en utilisant la méthode du cycle en V. Dès lors, pour limiter ces dangers, d'autres modèles de gestion de projet ont vu le jour.

Les méthodes agiles, apparues dans les années 1990, sont un groupe de pratiques de développement de projets informatiques, reposant sur une même philosophie.

Elles consistent à développer le produit de manière \textbf{itérative, incrémentale et adaptative} \figref{agiles}. Les fonctionnalités sont développées les unes après les autres, les plus importantes en premier. Le client est sollicité régulièrement afin de vérifier la conformité entre ses attentes et ce qui a été développé.

Le manifeste agile \appref{manifeste_agile} définit les méthodes agiles par ces quelques principes.\\

Extrait de \textit{Agile Manifesto} :

\begin{Quote}
Nous découvrons comment mieux développer des logiciels par la pratique et en aidant les autres à le faire.
Ces expériences nous ont amenés à valoriser :

\begin{itemize}
\item \textbf{Les individus et leurs interactions}, plus que les processus et les outils ;
\item \textbf{Des logiciels opérationnels}, plus qu’une documentation exhaustive ;
\item \textbf{La collaboration avec les clients}, plus que la négociation contractuelle ;
\item \textbf{L’adaptation au changement}, plus que le suivi d’un plan
\end{itemize}

Nous reconnaissons la valeur des seconds éléments, mais privilégions les premiers.
\end{Quote}

\fig{img/agiles_all.png}{agiles}{Représentation graphique des méthodes traditionnelles (à gauche) face aux méthodes agiles (à droite).}

En outre, les méthodes agiles permettent :

\begin{itemize}
\item une bonne conformité entre les attentes du client et le produit développé ;
\item une grande réactivité à ses demandes, même pendant la réalisation du produit ;
\item d'obtenir un produit partiellement fonctionnel, quelque soit l'état d'avancement du projet.
\end{itemize}

Ces méthodes font participer le client de manière bien plus intensive et régulière que la méthode du cycle en V. Cette particularité est donc à prendre en compte avec attention lors du choix de ces méthodes, et donc l'adapter en fonction du client. En effet certains clients peuvent ne pas apprécier d'être sollicités fréquemment et aimeraient que le fournisseur travaille avec plus d'autonomie, en se tenant au cahier des charges qu'il lui a fourni. On peut ainsi dire que les méthodes agiles, en plus d'être des méthodes de travail, sont des méthodes de collaboration entre le client et le fournisseur.

Il existe plusieurs méthodes agiles, qui ont leurs propres spécifications et qui doivent être adaptées en fonction du contexte. Ainsi, il ne suffit pas de choisir de développer un projet au moyen des méthodes agiles, il faut également se pencher sur la méthode à adopter. Parmi les plus connues, on retrouve la méthode Scrum et la méthode XP, que nous allons détailler dans ce chapitre.

\subsection{La méthode Scrum}

Le terme \textit{Scrum} (signifiant « mêlée ») provient du rugby à XV, sport qui a pour objectif d'atteindre un but grâce à une équipe soudée.
Le projet est découpé en plusieurs phases appelées \textit{sprints}, d'une durée d'environ un mois, pendant lesquelles l'équipe a comme objectif de développer un ensemble précis de fonctionnalités.
Au début du sprint, l'équipe se rassemble pour une \textit{réunion de planification de sprint}, pendant laquelle l'équipe sélectionne les tâches qui seront développées durant le prochain sprint, estiment leur temps de développement et se mettent d'accord sur le répartition du travail.

\begin{app}
Dans l'entreprise GFI, même si la méthode du cycle en V est majoritairement instaurée, certaines jeunes équipes travaillent avec la méthode agile Scrum. Elles utilisent une astuce à la fois ludique et efficace pendant les réunions de planification de sprint, appelée \textit{planning pocker}, pour estimer le temps de développement de chaque tâche.\\
Chaque membre de l'équipe dispose de plusieurs jetons représentant une certaine durée (1 jour, 2 jours, une semaine, etc.). Le Scrum Master énonce une tâche a effectuer et chaque membre choisit secrètement un jeton correspondant à la durée estimée de la tâche. Lorsque tous les membres de l'équipe ont sélectionné leur jeton, ils les disposent simultanément au centre de la table. L'équipe se base ensuite sur la valeur des différents jetons pour estimer la durée de la tâche et chaque personne est invitée à argumenter son choix.\\
Cette méthode assez originale favorise également la bonne ambiance au sein de l'équipe, ce qui impacte directement sur l'efficacité de la collaboration comme nous l'avons vu dans la première partie de ce dossier.
\end{app}

À la fin du sprint, l'équipe se retrouve pour une réunion appelée \textit{revue de sprint}, pour valider ensemble les fonctionnalités développées.
Pour davantage de dynamisme, la mêlée quotidienne (Daily Scrum) permet également aux développeurs de faire un point de coordination d'environ 15 minutes sur les tâches en cours et sur les difficultés rencontrées.

\fig{img/scrum.jpg}{scrum}{Cycle de développement d'un produit au moyen de la méthode Scrum.}

Les fonctionnalités sont souvent représentées physiquement au moyen de \textit{Post-its} à coller sur un tableau composé de trois colonnes correspondant à leur état d'avancement : \textit{à faire}, \textit{en cours} et \textit{réalisé}. Au sein de l'équipe, une personne désignée \textit{ScrumMaster} a pour mission de motiver l'équipe, de vérifier les tâches développées et proposer celles du prochain sprint. Ce statut n'a aucune signification hiérarchique et un nouveau ScrumMaster peut être désigné à chaque sprint.

\begin{app}
Dans l'entreprise GFI, on peut identifier assez rapidement les équipes qui travaillent avec la méthode Scrum car une partie des murs de leurs bureaux sont dédiés à l'utilisation des post-its. Du fait du principal client Airbus, les colonnes représentant les états des tâches sont accompagnée par l'image d'un avion au décollage, celle d'un avion en vol et celle d'un avion à l'atterrissage. Il s'agit d'un moyen mnémotechnique amusant pour signifier respectivement les 3 états : \textit{à faire}, \textit{en cours} et \textit{réalisé}.
\end{app}

La méthode Scrum facilite le dynamisme et le travail d'équipe, elle est adaptée aux projets qui peuvent évoluer.

Elle répond aux besoins d'évolution des projets car le client teste régulièrement la partie fonctionnelle du produit et en donne son avis au minimum à chaque sprint : ainsi il met en avant les problèmes et les amélioration attendues de manière régulière.

\subsection{La méthode XP (Extreme programming)}

La méthode XP consiste en différents principes à appliquer pendant la réalisation d'un projet. Ces derniers existent depuis de nombreuses années, toutefois ils sont ici poussés à l'extrême.

Extrait de \textit{Extreme Programming Explained} :
\begin{Quote}
\begin{itemize}
\item Puisque la revue de code est une bonne pratique, elle sera faite en permanence (par un binôme) ;
\item puisque les tests sont utiles, ils seront faits systématiquement avant chaque mise en œuvre ;
\item puisque la conception est importante, elle sera faite tout au long du projet (refactoring) ;
\item puisque la simplicité permet d'avancer plus vite, nous choisirons toujours la solution la plus simple ;
\item puisque la compréhension est importante, nous définirons et ferons évoluer ensemble des métaphores ;
\item puisque l'intégration des modifications est cruciale, nous l'effectuerons plusieurs fois par jour ;
\item puisque les besoins évoluent vite, nous ferons des cycles de développement très rapides pour nous adapter au changement.
\end{itemize}
\end{Quote}

Cette méthode repose sur des cycles de développement rapides, elle est adaptée aux équipes réduites avec des besoins qui peuvent évoluer.\\

Nous avons vus les avantages de l'utilisation de la méthode du cycle en V, mais les lourds inconveignants qu'elle apporte n'en fait pas une méthode toujours efficace en gestion de projet. Les méthodes agiles, plus modernes, semblent en revanche éviter de nombreuses difficultés, ce qui en ferait aujourd'hui un excellent choix pour démarrer sur un nouveau projet.

\section{Les normes}

Afin de bénéficier d'une bonne organisation entre les différentes équipes, ou entre le client et le fournisseur, il peut être intéressant de définir des normes de développement.

Ces normes de développement sont souvent impactées par le cycle de vie du logiciel et l'incidence qu'entraînera une éventuelle anomalie.

Pour un grand nombre d'application, comme les logiciels bureautiques ou les jeux vidéos, le logiciel sera en service quelques années avant que le client passe a la version supérieure. D'autre part, un bug aura un effet indésirable pour client mais n'aura pas une grande incidence.

En revanche, certains secteurs d'activité comme l'aéronautique sont assez particuliers puisque d'une part une anomalie informatique peut avoir d'importantes conséquences (jusqu'au crash d'un avion), d'autre part le cycle de vie est très long : la conception elle-même de l'avion peut prendre de nombreuses années, et l'avion sera en service encore pendant plusieurs décennies.

Il donc est nécessaire de communiquer avec les prochaines générations de développeurs qui vont maintenir le logiciel, notamment en laissant une documentation technique comme nous l'avons évoqué précedemment. Ainsi, dans ces situations, la collaboration se fait également sur le plan temporel.

\begin{app}

L'entreprise GFI comprend un \textit{Service Qualité}. Son but est de contrôler chaque livraison au quotidien afin de déterminer si les règles de qualité sont appliquées au sein de chaque projet. Par ailleurs elle met en place des plans et des processus à appliquer, de manière à harmoniser les différents projets GFI.

Une norme qualité est souvent nécessaire pour répondre aux appels d'offre, la plus utilisée étant la \textit{CAGR 3.1}.

Ces normes imposées varient en fonction du client. Dans le cas d'Airbus, elles varient même en fonction du service, car chacun a différentes exigences : un projet industriel, dont le code sera intégré à l'avion, mettra en application des processus assez lourds, avec une documentation importante, de nombreux tests et procédés de validation, tandis qu'un projet de recherche sera bien plus léger.

En revanche, les équipes de recherche ne se conforment pas à ces normes et leurs projets ne comportent pas de Service Qualité. En effet le code fournit par ces équipes a pour objectif de valider la conception et le fonctionnement d'un produit et d'obtenir un résultat au plus vite. Il ne sera pas intégré à l'application, mais sera repris par une autre équipe, qui développera la version industrielle en respectant les normes en vigueur.
\end{app}

\section{Conclusion}

On constate malgré tout que ce n'est pas parce qu'une méthode est en apparence plus moderne et offre de meilleurs avantages, qu'elle sera utilisée et acceptée par tous. Les méthodes agiles présentent certes une grande évolution, mais les avantages offerts doivent être mesurés en fonction du contexte, notamment les habitudes de fonctionnement au sein de l'entreprise et celles du client, comme nous le verrons dans le prochain chapitre.

\begin{app}
Dans l'entreprise GFI, qui se veut innovante (son slogan est \textit{New challenges, new ideas}), la méthode du cycle en V est malgré tout utilisée dans la plupart des projets. Cette entreprise n'est pas jeune et au fil des années certaines habitudes de fonctionnement se sont instaurées, notamment sur les méthodes de travail. La raison principale pour laquelle GFI utilise la méthode du cycle en V est qu'elle est très accoutumée à celle-ci : elle en connaît les défaut et sait les prévoir, les contourner et les affronter. Les dangers que peut entraîner l'utilisation de cette méthode seront donc moins nombreux et auront de moins grandes conséquences.
\end{app}

Changer une méthode qui fait partie des habitudes d'une équipe pourrait ainsi entraîner des oppositions et des conflits de la part des salariés. On peut ajouter à cela le temps de formation nécessaire à l'adoption d'une nouvelle méthode, qui est non négligeable à la fois sur le plan temporel et financier. La loi de Brooks étudiée précédemment peut ainsi s'appliquer en partie aux méthodes de travail : le temps et le prix de formation et d'adaptation pour instaurer une méthode plus moderne peuvent être supérieurs aux avantages que cette méthode aurait apporté.

En outre, certains pourrons donc se fier à l'idée qu'il ne sert à rien ne modifier une manière de fonctionner si celle-ci est déjà fonctionnelle et ne présente pas de problèmes particuliers.\\

Nous avons vu les différentes méthodes de travail pouvant être adoptées pour la réalisation d'un projet informatique. Nous avons également vu comment le rapport avec le client et les préférences au sein de l'entreprise peuvent influencer le choix de ces méthodes. Nous allons donc maintenant étudier les différents outils favorisant le travail collaboratif.
